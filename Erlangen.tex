\documentclass[aspectratio=169]{beamer}
\usepackage{amsmath,amssymb}
\usepackage{mathtools}
\usepackage{euler}
\usepackage{enumitem}
\usepackage{appendixnumberbeamer}
\usepackage{fontawesome5}
\usefonttheme{professionalfonts}
\setbeamertemplate{navigation symbols}{}

\usepackage{tikz}
\usetikzlibrary{calc}
\usepackage{tikz-cd}

\tikzset{
  curarrow/.style n args={1}{
    rounded corners=8pt,
    to path={
      -- ([xshift=-50pt]\tikztostart.center)
      |- (#1) node[fill=white,inner sep=1pt] {$\scriptstyle d_*$}
      -| ([xshift=50pt]\tikztotarget.center)
      -- (\tikztotarget)
      \tikztonodes
    }
  }
}

\usepackage{mymacro}
\usepackage{tensor}
\usepackage[
  backend=biber,
  style=authoryear,
  citestyle=authoryear,
  natbib=true,
  giveninits=true,
  uniquename=init,
  maxcitenames=2,
  mincitenames=1,
  maxbibnames=99,
  date=year,
  doi=true,
  url=false,
  isbn=false,
  eprint=true,
  sorting=nyt,
  sortcites=true,
  dashed=false
]{biblatex}

\setbeamertemplate{bibliography item}{
  \textcolor{structure.fg}{\faBook}
}
\addbibresource{Slides.bib}

\renewcommand{\ldots}{
  \mathinner{{\ldotp}{\ldotp}}
}
\newcommand{\G}{\mathcal{G}}
\newcommand{\K}{\mathcal{K}}
\newcommand{\Hh}{\mathcal{H}}
\newcommand{\Gz}{\G^{(0)}}
\newcommand{\red}[2]{#1\!\mid_{#2}}
\newcommand{\ZZ}{\mathbb{Z}}
\newcommand{\Hom}{\operatorname{Hom}}
\newcommand{\Ob}{\operatorname{Ob}}
\newcommand{\Top}{\operatorname{Top}}
\newcommand{\supp}{\operatorname{supp}}
\newcommand{\fprod}[3]{#1\,\tensor[_#2]{\times}{_{#3}}\,}
\newcommand{\stimesr}{\tensor[_s]{\times}{_r}}
\newcommand{\Gstimesr}{\tensor[_{s_{\mathcal{G}}}]{\times}{_{r_{\mathcal{G}}}}}
\newcommand{\Hstimesr}{\tensor[_{s_{\mathcal{H}}}]{\times}{_{r_{\mathcal{H}}}}}
\newcommand{\ptimesr}{\tensor[_p]{\times}{_r}}

\usepackage{silence}
\WarningFilter{latex}{}
\WarningFilter{transparent}{Loading aborted, because pdfTeX is not running in PDF mode}
\WarningFilter{inputenc}{inputenc package ignored with utf8 based engines}
\WarningFilter{latexfont}{}
\WarningFilter{hyperref}{}
\WarningFilter{biblatex}{}
\WarningFilter{biblatex}{Please (re)run Biber on the file}
\WarningFilter{biblatex}{The following entry could not be found}
\WarningFilter{biblatex}{Patching footnotes failed}
\WarningFilter{pdftex}{}
\WarningFilter{luatex}{}
\WarningFilter{rerunfilecheck}{}
\WarningFilter{microtype}{}

\hbadness=10000
\vbadness=10000
\hfuzz=1000pt
\vfuzz=1000pt
\setlength{\emergencystretch}{2em}

\makeatletter
\def\@latex@warning#1{}%
\def\@latex@warning@no@line#1{}%
\makeatother

\definecolor{gold}{RGB}{184,134,11}

\setbeamertemplate{footline}{
  \raisebox{10pt}{
    \makebox[\paperwidth]{
      \hfill
      \makebox[7em]{\normalsize\texttt{\insertframenumber/\inserttotalframenumber}}
    }
  }
}

\title{\textbf{Universal Coefficient Theorem} \\ \textbf{Moore--Mayer--Vietoris Sequence} \\ for the Homology of Ample Groupoids}
\author{$\highlight[blue]{\text{Luciano Melodia}}$ {\scriptsize M.A., B.Sc., B.A.}\\ Application to Friedrich-Alexander-Universität Erlangen--Nürnberg\\ Chair of Theoretical Computer Science}
\date{3 March 2026}

\begin{document}

\begin{frame}[plain]
  \maketitle
\end{frame}

\begin{frame}{Topics}{What will we cover today?}
  \highlight[blue]{\text{Homology of ample groupoids.}} \\
  {\scriptsize Homology via the Moore chain complex of ample groupoids.}\\[0.3cm]
  \highlight[blue]{\text{Universal coefficient theorem.}} \\
  {\scriptsize A universal coefficient theorem for discrete abelian groups.}\\[0.3cm]
  \highlight[blue]{\text{Moore--Mayer--Vietoris sequence.}} \\
  {\scriptsize A Mayer--Vietoris type sequence for clopen saturated covers.}
\end{frame}

\begin{frame}[plain,c]
  \begin{center}
    \Huge \textbf{Homology of Ample Groupoids}
  \end{center}
\end{frame}

\begin{frame}{Standing Hypotheses}{We investigate ample groupoids.}
  We consider \(C_c(\highlight[blue]{\mathcal{G}_n},\highlight[red]{A})\).

  \(\highlight[red]{A}\) is a topological abelian group.

  \(\highlight[blue]{\mathcal{G}}\) is an ample groupoid.

  \(\highlight[blue]{\mathcal{G}_n}\) is the space of \(n\)\nobreakdash-multiplicable arrows.

  \vspace{1em}
  \pause
  \(\highlight[blue]{\text{Why is this important?}}\)
  \vspace{0.2em}
  \begin{itemize}[noitemsep,nolistsep]
    \item
      \(\highlight[blue]{\G \ \text{\'etale:}}\) structure maps in the nerve, such as face maps \(d_i\) and degeneracies \(s_j\), are local homeomorphisms, so pushforwards \((d_i)_*\) are defined by finite fibre sums on compact support.\pause
    \item
      \(\highlight[blue]{\G \ \text{ample:}}\) compact open bisections form a basis.

      \(C_c(\G,\ZZ)\) is generated by \(\chi_K\) for compact open sets \(K\).
  \end{itemize}
\end{frame}

\begin{frame}{The Nerve}{On what do we compute homology?}
  \(\G_\bullet\coloneqq (\G_n,(d_i)_{i=0}^n,(s_j)_{j=0}^n)_{n\ge 0}\) is a \(\highlight[red]{\text{simplicial space}}\).
  \pause

  \(\highlight[blue]{\text{Face maps.}}\)
  \(d_i\colon \G_n\to \G_{n-1}\), \(n=1\): \(d_0=s\), \(d_1=r\). For \(n\ge 2\):
  \(
    d_i(\mathbf{g})\coloneqq
    \begin{cases}
      (g_2,\ldots,g_n), & i=0,\\
      (g_1,\ldots,g_i\cdot g_{i+1},\ldots,g_n), & 1\le i\le n-1,\\
      (g_1,\ldots,g_{n-1}), & i=n.
    \end{cases}
  \)
  \pause
  \vspace{0.2cm}

  \(\highlight[blue]{\text{Degeneracy maps.}}\)
  \(s_j\colon \G_n\to \G_{n+1}\), \(n\ge 0\):
  \vspace{-0.3em}
  \(
  \begin{aligned}
    s_j(\mathbf{g}) &\coloneqq \begin{cases}
      u(x), & n=0,\ x\in \G_0,\\
      (u(r(g_1)),g_1,\ldots,g_n), & n\ge 1,\ j=0,\\
      (g_1,\ldots,g_j,\,u(r(g_{j+1})),\,g_{j+1},\ldots,g_n), & n\ge 2,\ 1\le j\le n-1,\\
      (g_1,\ldots,g_n,\,u(s(g_n))), & n\ge 1,\ j=n.
    \end{cases}\\
   \G_n &\coloneqq
    \begin{cases}
      \G_0, & n=0,\\
      \{\mathbf{g} \in \G^{\,n}\mid s(g_i)=r(g_{i+1})\ \text{for}\ 1\le i<n\}, & n\ge 1.
    \end{cases}
  \end{aligned}
  \)
\end{frame}

\begin{frame}{Moore Chains and Boundary}{Compactly supported chains on the nerve.}
  \(\highlight[blue]{\text{Chains.}}\)
  \(C_c(\G_n,A)\) denotes the abelian group of continuous maps \(f\colon \G_n\to A\) with \(\highlight[gold]{\text{compact support}}\).
  If \(A\) is \(\highlight[red]{\text{discrete}}\), then every \(f\in C_c(\G_n,A)\) is \(\highlight[gold]{\text{locally constant}}\).
  \pause
  \vspace{0.3cm}

  \(\highlight[blue]{\text{Boundary.}}\)
  Since \(\G\) is \'etale, each face map \(d_i\colon \G_n\to \G_{n-1}\) is a local homeomorphism, hence the pushforward is well-defined:
  \[
    \begin{aligned}
      (d_i)_* &: C_c(\G_n,A)\to C_c(\G_{n-1},A),
      \qquad
      (d_i)_*f(y)\coloneqq \sum_{x\in d_i^{-1}(y)\cap \supp(f)} f(x),\\[-1.2em]
      \partial_n &\coloneqq \sum_{i=0}^n (-1)^i (d_i)_*
      \colon C_c(\G_n,A)\to C_c(\G_{n-1},A).
    \end{aligned}
  \]
\end{frame}

\begin{frame}[plain,c]
  \begin{center}
    \Huge \textbf{Universal Coefficient Theorem}
  \end{center}
\end{frame}

\begin{frame}{Universal Coefficient Theorem}{Is \(\ZZ\) enough to recover homology through \(A\)?}
  \(\G\) ample groupoid.\\
  \(A\) discrete abelian group.\\\pause
  For all \(n\ge 0\) there is a natural short exact sequence in \(\mathbf{Ab}\):
  \[
    0
    \to H_n(\G)\otimes_{\ZZ} A
    \xrightarrow{\highlight[red]{\iota_n^{\G}}}
    H_n(\G;A)
    \xrightarrow{\highlight[blue]{\kappa_n^{\G}}}
    \operatorname{Tor}_1^{\ZZ}\bigl(H_{n-1}(\G),A\bigr)
    \to 0.
  \]
  The sequence splits, though not canonically:
  \[
    H_n(\G;A)
    \cong
    \bigl(H_n(\G)\otimes_{\ZZ} A\bigr)
    \oplus
    \operatorname{Tor}_1^{\ZZ}\bigl(H_{n-1}(\G),A\bigr).
  \]
\end{frame}

\begin{frame}{Universal Coefficient Theorem}{If \(A\) is discrete, then \(C_c(\G_n,A)\) is a free \(A\)-module.}
  Let \(f\in C_c(\G_n,A)\). Since \(A\) is discrete and \(\supp(f)\) is compact, \(\operatorname{im}(f)=\{a_1,\ldots,a_m\}\) is finite.
  Set \(K_i\coloneqq f^{-1}(\{a_i\})\) for \(1\le i\le m\).\pause
  
  Each \(K_i\) is clopen in \(\G_n\), the sets \(K_i\) are pairwise disjoint, and \(\supp(f)=\bigsqcup_{i=1}^m K_i\).
  \[
    f=\sum_{i=1}^m a_i\,\chi_{K_i}\qquad\text{in }C_c(\G_n,A).
  \]
  \pause
  Therefore the canonical map
  \[
    \bigoplus_{K\in \mathcal{K}(\G_n)} A \longrightarrow C_c(\G_n,A),
    \qquad
    (a_K)_{K\in \mathcal{K}(\G_n)}\longmapsto \sum_{K\in \mathcal{K}(\G_n)} a_K\,\chi_K,
  \]
  is surjective.
  It is injective: if \(\sum_{i=1}^m a_i\,\chi_{K_i}=0\) with \(K_i\) pairwise disjoint compact open, then evaluating at any \(g\in K_i\) gives \(a_i=0\).
  Thus
  \[
    C_c(\G_n,A)\cong \bigoplus_{K\in \mathcal{K}(\G_n)} A,
    \quad \highlight[blue]{\text{free as an \(A\)\nobreakdash-module}}.
  \]
\end{frame}

\begin{frame}{Non-discrete Coefficients: What Fails}{The general result.}
For \(X\) $\highlight[blue]{\text{locally compact}}$, $\highlight[blue]{\text{totally disconnected}}$, $\highlight[blue]{\text{Hausdorff}}$ with a $\highlight[blue]{\text{basis of compact open sets}}$ and an abelian group \(A\), consider the canonical map
\[
\Phi_X\colon C_c(X,\ZZ)\otimes_{\ZZ}A \to C_c(X,A),
\qquad
\chi_U\otimes a \mapsto a\,\chi_U.
\]
Then
\[
\operatorname{im}(\Phi_X)\subseteq \{\xi\in C_c(X,A)\mid \xi(X)\ \text{is finite}\}.
\]
\pause

In particular, \(\Phi_X\) may $\highlight[red]{\text{fail}}$ to be surjective for $\highlight[red]{\text{non-discrete } A}$.
Moreover,
\[
\Phi_X\ \text{surjective}
\Leftrightarrow
\forall\,\xi\in C_c(X,A):\ \xi(X)\ \text{finite}
\Leftrightarrow
\Phi_X\ \text{is an isomorphism}.
\]
\pause

If \(A\) is discrete, then \(\Phi_X\) is an isomorphism.
The converse can fail.
\end{frame}

\begin{frame}[plain,c]
  \begin{center}
    \Huge \textbf{Moore-Mayer-Vietoris Sequence}
  \end{center}
\end{frame}

\begin{frame}{Mayer-Vietoris vs. Moore-Mayer-Vietoris}{From saturated covers to homology.}
  \begin{columns}[t]
    \begin{column}{0.483\hsize}
      \(\highlight[red]{\textbf{Mayer--Vietoris:}}\)
      \vspace{0.2cm}

      \(
      \begin{aligned}
      &X = U_1\cup U_2, \\ &U_1,U_2 \subseteq X \text{ open}.
      \end{aligned}
      \)
      \vspace{0.3cm}

      \(\highlight[blue]{\text{Good cover:}}\)

      \(U_1,\ U_2,\ U_1\cap U_2\).\\[0.4em]

      \vspace{0.6cm}

      \(
      \highlight[gold]{\text{Compute \(H_\bullet(X)\)} from:}
      \)

      \(H_\bullet(U_1)\), \(H_\bullet(U_2)\), \(H_\bullet(U_1\cap U_2)\).
    \end{column}
    \begin{column}{0.517\hsize}
      \(\highlight[red]{\textbf{Moore--Mayer--Vietoris:}}\)
      \vspace{0.2cm}
      \pause

      \(
      \begin{aligned}
      &\G_0=U_1\cup U_2,\\
      &U_1,U_2 \subseteq \G_0 \ \text{clopen.}
      \end{aligned}
      \)
      \vspace{0.3cm}
      \pause

      \(\highlight[blue]{\text{Saturated cover:}}\)

      \(\forall \ x\in U:\ s(g)=x \Rightarrow r(g)\in U\).\\[0.4em]
      \(\G|_U \coloneqq \{g\in \G \mid s(g),r(g)\in U\}\),
      
      \(\G|_U\hookrightarrow \G\) open.

      \vspace{0.6cm}
      \pause

      \(\highlight[gold]{\text{Compute \(H_\bullet(\G)\) from:}}\)

      \(H_\bullet(\G|_{U_1})\), \(H_\bullet(\G|_{U_2})\), \(H_\bullet(\G|_{U_1\cap U_2})\).
    \end{column}
  \end{columns}
\end{frame}

\begin{frame}{Reductions and Moore--Mayer--Vietoris}{Long Natural Moore-Mayer-Vietoris Sequence for Homology.}
For \(U\subseteq \G_0\) define the reduction
\[
\G|_U \coloneqq \{g\in \G \mid s(g),r(g) \in U\},
\qquad
(\G|_U)_0 = U,
\]
with structure maps the restrictions of \(u,m,s,r,-^{-1}\) to \(\G|_U\).
For \(i\in\{1,2\}\) and \(U_{12}\coloneqq U_1\cap U_2\) we write
\(\G|_{U_i}\) and \(\G|_{U_{12}}\).

\vspace{0.3cm}
\pause

\begin{center}
$\highlight[blue]{\textbf{Moore--Mayer--Vietoris long exact homology sequence:}}$
\vspace{0.3cm}

\includegraphics[width=13cm]{mvs.png}
\end{center}
\end{frame}

\begin{frame}{Takeaways}{What you get and how to use it.}
\(\highlight[gold]{\text{Setting:}}\)
\(\G\) ample \'etale, \(A\) a discrete abelian group,

Moore chains \(C_c(\G_n,A)\) with boundary \(\partial=\sum_{i=0}^n (-1)^i(d_i)_*\).

\vspace{0.3cm}

\begin{center}
    \(\highlight[blue]{\text{Two structural tools:}}\)
\end{center}
\pause

\(\highlight[red]{\text{UCT for discrete coefficients:}}\)
\(
    0\to H_n(\G)\otimes_{\ZZ}A
    \xrightarrow{\iota_n^{\G}}
    H_n(\G;A)
    \xrightarrow{\kappa_n^{\G}}
    \operatorname{Tor}_1^{\ZZ}\bigl(H_{n-1}(\G),A\bigr)\to 0.
\)
\vspace{0.2cm}
\pause

\(\highlight[red]{\text{Moore--Mayer--Vietoris}}\)

for a \(\highlight[blue]{\text{clopen saturated cover}}\)
\(U_1\cup U_2=\G_0\).

There is a natural long exact sequence relating

\[
H_\bullet(\G;A), H_\bullet(\G|_{U_1};A), H_\bullet(\G|_{U_2};A), H_\bullet(\G|_{U_{12}};A).
\]
\end{frame}

\begin{frame}{Takeaways}{What you get and how to use it.}
\begin{center}
\(\highlight[blue]{\text{Why discreteness matters:}}\)
\end{center}

For non-discrete \(A\), the canonical comparison
\[
\Phi_X:\ C_c(X,\ZZ)\otimes_{\ZZ}A \to C_c(X,A),
\qquad
\chi_U\otimes a \mapsto a\,\chi_U,
\]
need not be surjective. Tensor-level reduction in UCT can fail.

\vspace{0.1cm}
\pause

\begin{center}
\(\highlight[blue]{\text{How to use in practice:}}\)
\end{center}

Choose a clopen saturated cover.

\(U_1,U_2\subseteq \G_0\) so that reductions \(\G|_{U_1}\), \(\G|_{U_2}\), \(\G|_{U_{12}}\) are computable.

Compute integral homology.

\(H_\bullet(\G|_{U_i})\), \(H_\bullet(\G|_{U_{12}})\),
then \(\highlight[red]{\text{glue}}\) to \(H_\bullet(\G)\) via MMV.
\end{frame}

\begin{frame}[plain,c]
  \begin{center}
    \Huge \textbf{Thank you.}
  \end{center}
\end{frame}

\appendix

\begin{frame}[plain,c]
  \begin{center}
    \Huge \textbf{Homology of SFT Groupoids}
  \end{center}
\end{frame}

\begin{frame}{Example: Diaconu-Renault Groupoid}{Computing Homology with Moore--Mayer--Vietoris + UCT.}
  \(A\in \operatorname{Mat}(N\times N, \mathbb N_0)\) with no zero row and no zero column.\\
  \((E^{(1)}_A,E^{(0)}_A,s_{E^{(1)}_A},r_{E^{(1)}_A})\) a finite directed graph whose adjacency matrix is \(A\).\\[0.2cm]
  The $\highlight[red]{\text{infinite path space}}$ is given by $\highlight[blue]{\text{Sims 2021, 2.5:}}$\\
  \(E_A^\infty
  = \left\{(e_n)_{n\ge 1}\in (E_A^1)^{\mathbb N}\ \middle|\ r_{E^{(1)}_A}(e_n)=s_{E^{(1)}_A}(e_{n+1})\ \text{for all } n\ge 1 \right\}
  \)
  with \\
  \(\sigma\colon E_A^{\infty}\to E_A^{\infty}, (e_0,e_1,e_2,\ldots)\mapsto (e_1,e_2,e_3,\ldots)\).\\
  \((\mathcal G_A)_0=E_A^\infty\). \\
  \(
    \begin{aligned}
      (\mathcal G_A)_1 = \{(x,n,y)\in E_A^\infty\times\mathbb Z\times E_A^\infty \mid \exists\ k, \ell\in\mathbb N_0:\ n=k-\ell, \ \sigma^k(x)=\sigma^\ell(y)\}.
    \end{aligned}
  \)
  \(s(x,n,y)=y\), \(r(x,n,y)=x\), \(1_x=(x,0,x)\),\\
  \((x,n,y)^{-1}=(y,-n,x)\), \((x,n,y) \cdot (y,m,z)=(x,n+m,z)\) if \(s(x,n,y)=r(y,m,z)\).
\end{frame}

\begin{frame}{Example: Diaconu-Renault Groupoid}{Homology of SFT-Groupoids is well known.}
  \(\mathbf{1}-A^{\mathsf T}\) acts on \(\mathbb Z^N\) and we have by $\highlight[blue]{\text{Matui 2012, 4.14:}}$
  \[
    \begin{aligned}
      H_0(\mathcal G_A) &\cong \highlight[red]{\operatorname{coker}(\mathbf{1}-A^{\mathsf T})},\\
      H_1(\mathcal G_A) &\cong \highlight[red]{\ker(\mathbf{1}-A^{\mathsf T})},\\
      H_n(\mathcal G_A) &= 0\ \ \text{for}\ n\ge 2.
    \end{aligned}
  \]
  Consider now:
  \[
    A=\begin{pmatrix}2&1\\[2pt]1&0\end{pmatrix},\qquad
    B=\begin{pmatrix}2&1\\[2pt]1&2\end{pmatrix},\qquad
    C=(3).
  \]
  We compute the integral homology of \(\mathcal G_A\), \(\mathcal G_B\), and \(\mathcal G_C\).
\end{frame}

\begin{frame}{Example: Diaconu-Renault Groupoid}{Computing Homology for $\G_A$.}
  For \(A\) we have
  \[
    \mathbf{1}-A^{\mathsf T}=
    \begin{pmatrix}-1&-1\\[2pt]-1&1\end{pmatrix},
    \qquad
    \det(\mathbf{1}-A^{\mathsf T})=-2.
  \]
  Hence \(\mathbf{1}-A^{\mathsf T}\) has full rank over \(\mathbb Z\) and \(\highlight[blue]{\ker(\mathbf{1}-A^{\mathsf T})=0}\).

  Moreover, the Smith normal form is
  \[
    \begin{pmatrix}
      -1 & -1\\
      -1 &  1
    \end{pmatrix}
    \overset{R_1\leftarrow - R_1}{\longrightarrow}
    \begin{pmatrix}
      1 & 1\\
      -1 & 1
    \end{pmatrix}
    \overset{R_2\leftarrow R_2+R_1}{\longrightarrow}
    \begin{pmatrix}
      1 & 1\\
      0 & 2
    \end{pmatrix}
    \overset{C_2\leftarrow C_2- C_1}{\longrightarrow}
    \begin{pmatrix}
      1 & 0\\
      0 & 2
    \end{pmatrix},
  \]
  so \(\highlight[red]{\operatorname{coker}(\mathbf{1}-A^{\mathsf T})\cong \mathbb Z/2\mathbb Z}\).\\[0.2cm]
  We get \(\highlight[red]{H_0(\mathcal G_A) \cong \mathbb Z/2\mathbb Z}\), \(\highlight[blue]{H_1(\mathcal G_A) = 0}\), \(\highlight[gold]{H_n(\mathcal G_A) = 0} \ \text{for}\ n\ge 2.\)
\end{frame}

\begin{frame}{Example: Diaconu-Renault Groupoid}{Computing Homology for $\G_B$.}
  For \(B\) we have
  \[
    \mathbf{1}-B^{\mathsf T}=
    \begin{pmatrix}-1&-1\\[2pt]-1&-1\end{pmatrix}.
  \]
  \((\mathbf{1}-B^{\mathsf T})(x,y)^{\mathsf T}=0 \Leftrightarrow -x-y=0\), hence \(\highlight[blue]{\ker(\mathbf{1}-B^{\mathsf T})\cong\mathbb Z}\).\\
  The image is generated by \((1,1)\), which is primitive in \(\mathbb Z^2\), so \(\highlight[red]{\operatorname{coker}(\mathbf{1}-B^{\mathsf T})\cong \mathbb Z^2/\langle(1,1)\rangle_{\mathbb Z}\cong\mathbb Z}\).\\[0.2cm]

  Thus, we have for homology \(\highlight[red]{H_0(\mathcal G_B) \cong \mathbb Z}\), \(\highlight[blue]{H_1(\mathcal G_B) \cong \mathbb Z}\), \(\highlight[gold]{H_n(\mathcal G_B) = 0} \ \text{for}\ n\ge 2.\)
\end{frame}

\begin{frame}{Example: Diaconu-Renault Groupoid}{Computing Homology for $\G_C$.}
  For \(C\) we have \(\mathbf{1}-C^{\mathsf T}=-2\), so\\
  \(\highlight[blue]{\ker(\mathbf{1}-C^{\mathsf T})=0}\) and \(\highlight[red]{\operatorname{coker}(\mathbf{1}-C^{\mathsf T})\cong \mathbb Z/2\mathbb Z}\).\\[0.2cm]

  Hence \(\highlight[red]{H_0(\mathcal G_C) \cong \mathbb Z/2\mathbb Z}\), \(\highlight[blue]{H_1(\mathcal G_C) = 0}\), \(\highlight[gold]{H_n(\mathcal G_C) = 0} \ \text{for}\ n\ge 2.\)
\end{frame}

\begin{frame}{Example: Diaconu-Renault Groupoid}{The disjoint union groupoid.}
We have $\G = \G_A \sqcup \G_B \sqcup \G_C$, the disjoint union groupoid.

The nerve decomposes levelwise to $\G_n = (\G_A)_n \sqcup (\G_B)_n \sqcup (\G_C)_n$.

The Moore chain complex splits as a direct sum, thus
\[
H_n(\mathcal G)\cong H_n(\mathcal G_A)\oplus H_n(\mathcal G_B)\oplus H_n(\mathcal G_C)
\ \text{for} \ n\ge 0.
\]
In particular
\[
\begin{aligned}
H_0(\mathcal G) &\cong \mathbb Z\oplus (\mathbb{Z}/2\mathbb{Z})^2, & H_1(\mathcal G) &\cong \mathbb Z, & H_n(\mathcal G) &= 0\ \text{for} \ n\ge 2.
\end{aligned}
\]
Define
\(
U_1 \coloneqq (\mathcal G_A)_0\sqcup(\mathcal G_B)_0, U_2 \coloneqq (\mathcal G_B)_0\sqcup(\mathcal G_C)_0.
\)
\end{frame}

\begin{frame}{Example: Diaconu-Renault Groupoid}{The reduction groupoids.}
The reductions are
\[
\begin{aligned}
\mathcal G|_{U_1} &= \mathcal G_A\sqcup \mathcal G_B,&
\mathcal G|_{U_2} &= \mathcal G_B\sqcup \mathcal G_C,&
\mathcal G|_{U_1\cap U_2} &= \mathcal G_B.
\end{aligned}
\]
This yields the long exact sequence
\[
\begin{aligned}
\cdots\to
H_n(\mathcal G_B)
&\xrightarrow{\highlight[red]{\alpha_n}}
H_n(\mathcal G_A\sqcup\mathcal G_B)\oplus H_n(\mathcal G_B\sqcup\mathcal G_C)
\xrightarrow{\highlight[blue]{\beta_n}} \\
&\xrightarrow{\highlight[blue]{\beta_n}}
H_n(\mathcal G)
\xrightarrow{\highlight[gold]{\partial_n}}
H_{n-1}(\mathcal G_B)
\to\cdots.
\end{aligned}
\]
\end{frame}


\begin{frame}{Example: Diaconu-Renault Groupoid}{Explicit formulas for \(\alpha_n\), \(\beta_n\), \(\partial_n\)}
\[
\begin{aligned}
\highlight[red]{\alpha_n}&\colon H_n(\G_B)\to
H_n(\G_A)\oplus H_n(\G_B)\oplus H_n(\G_B)\oplus H_n(\G_C),\\
&[b] \mapsto ([0],[b],[-b],[0]),\\
\highlight[red]{\beta_n}&\colon
H_n(\G_A)\oplus H_n(\G_B)\oplus H_n(\G_B)\oplus H_n(\G_C)
\to H_n(\G_A)\oplus H_n(\G_B)\oplus H_n(\G_C),\\
&([a],[b_1],[b_2],[c]) \mapsto ([a],[b_1+b_2],[c]).
\end{aligned}
\]

\vspace{0.25cm}

\(\highlight[gold]{\partial_n}\) vanishes in this example by exactness
\(
\highlight[gold]{\partial_n}=0\colon H_n(\G)\to H_{n-1}(\G_B),
\)
since \(\beta_n\) is surjective and
\(
\ker(\beta_n)=\{([0],[b],[-b],[0])\mid b\in H_n(\G_B)\}=\operatorname{im}(\alpha_n)
\).
\end{frame}

\begin{frame}{Finite coefficients via UCT}{Final homology groups for \(\ZZ/p\ZZ\).}
\small
Fix a prime \(p\). Assume
\(H_2(\G)=0, H_1(\G)\cong \ZZ, H_0(\G)\cong \ZZ\oplus (\ZZ/2\ZZ)^2.\)

\vspace{0.2cm}

\(\highlight[blue]{\text{Vanishing in higher degrees:}}\)
\(H_n(\G;\ZZ/p\ZZ)=0
\quad\text{for all } n\ge 2.
\)

\(\highlight[blue]{\text{Degree \(0\):}}\)
\[
H_0(\G;\ZZ/p\ZZ)\cong H_0(\G)\otimes_{\ZZ}\ZZ/p\ZZ
\cong
\begin{cases}
\ZZ/p\ZZ, & \text{for \(p\) odd},\\
(\ZZ/2\ZZ)^3, & \text{for \(p=2\)}.
\end{cases}
\]

\(\highlight[blue]{\text{Degree \(1\) via UCT:}}\)
\[
0\to H_1(\G)\otimes_{\ZZ}\ZZ/p\ZZ \to H_1(\G;\ZZ/p\ZZ)\to \operatorname{Tor}_1^{\ZZ}\bigl(H_0(\G),\ZZ/p\ZZ\bigr)\to 0,
\]
hence
\[
H_1(\G;\ZZ/p\ZZ)\cong
\begin{cases}
\ZZ/p\ZZ, & \text{for \(p\) odd},\\
(\ZZ/2\ZZ)^3, & \text{for \(p=2\)}.
\end{cases}
\]
\end{frame}

\begin{frame}[plain,c]
  \begin{center}
    \Huge \textbf{Proof of the UCT}
  \end{center}
\end{frame}

\begin{frame}{Proof of the UCT}{Step 1: Chain-level identification.}
Let \(f\in C_c(\G_n,A)\) and write \(\operatorname{im}(f)=\{a_1,\ldots,a_m\}\).
Set \(K_i\coloneqq f^{-1}(\{a_i\})\). Then \(\supp(f)=\bigsqcup_{i=1}^m K_i\) with \(K_i\) clopen and \(f|_{K_i}\equiv a_i\).

\vspace{0.15cm}

\(\highlight[red]{\text{Extension by \(0\):}}\) \(\chi_{K_i}\in C_c(\G_n,\ZZ)\) and
\[
f=\sum_{i=1}^m a_i\,\chi_{K_i}\qquad\text{in }C_c(\G_n,A).
\]

\vspace{0.15cm}
\pause

Define the canonical \(\ZZ\)-bilinear map
\[
\Phi_{\G_n}\colon C_c(\G_n,\ZZ)\otimes_{\ZZ}A \longrightarrow C_c(\G_n,A),
\qquad
\xi\otimes a \longmapsto a\cdot \xi.
\]
\pause
\[
\Phi_{\G_n}\ \text{is surjective, and injective since } C_c(\G_n,\ZZ)\ \text{is free on }\{\chi_K\mid K\in \mathcal K(\G_n)\}.
\]
\pause
\[
C_c(\G_n,\ZZ)\otimes_{\ZZ}A \cong C_c(\G_n,A)
\qquad \text{for discrete }A.
\]
\end{frame}

\begin{frame}{Proof of the UCT}{Step 2: Compatibility with the boundary.}
For each face map \(d_i\colon \G_n\to \G_{n-1}\), the pushforward \((d_i)_*\) is \(\ZZ\)-linear and satisfies
\[
(d_i)_*(\xi\cdot a)=\bigl((d_i)_*\xi\bigr)\cdot a
\qquad
\text{for } \xi\in C_c(\G_n,\ZZ),\ a\in A.
\]
\pause

Hence \(\Phi_{\G_\bullet}\) intertwines the Moore boundary:
\[
\Phi_{\G_{n-1}}\circ(\partial_n\otimes \mathrm{id}_A)=\partial_n\circ \Phi_{\G_n}.
\]
\pause

Therefore \(\Phi_{\G_\bullet}\) is an isomorphism of chain complexes
\[
C_c(\G_\bullet,\ZZ)\otimes_{\ZZ} A \cong C_c(\G_\bullet,A).
\]
\end{frame}

\begin{frame}{Proof of the UCT}{Step 3: Apply the classical algebraic UCT.}
The Moore complex \(C_c(\G_\bullet,\ZZ)\) is a chain complex of free abelian groups.
Applying the classical algebraic UCT to \(C_c(\G_\bullet,\ZZ)\) and transporting across the chain isomorphism from Steps 1--2 yields, for all \(n\ge 0\), a short exact sequence
\[
0
\to H_n(\G)\otimes_{\ZZ} A
\xrightarrow{\iota_n^{\G}}
H_n(\G;A)
\xrightarrow{\kappa_n^{\G}}
\operatorname{Tor}_1^{\ZZ}\bigl(H_{n-1}(\G),A\bigr)
\to 0.
\]
\pause

This sequence is natural in \(\G\) and in discrete \(A\).
In general, for non-discrete topological abelian groups \(A\), Moore homology need not satisfy such a UCT.
\end{frame}

\begin{frame}[plain,c]
  \begin{center}
    \Huge \textbf{Failure of Isomorphism}
  \end{center}
\end{frame}

\begin{frame}{Non-discrete Coefficients: What Fails}{Failure of the tensor comparison.}
If $\highlight[red]{\text{$A$ is non-discrete}}$ and $\highlight[red]{\text{$0\in A$ is not isolated}}$, then surjectivity of \(\Phi_X\) can fail even for compact, totally disconnected spaces with a basis of clopen subsets.
Set
\[
\begin{aligned}
X &\coloneqq \left\{\sum_{n=1}^\infty \frac{a_n}{3^n}\ \middle|\ a_n\in\{0,2\}\right\}\subset [0,1],\\
A &\coloneqq (\mathbb{R},\mathcal{O}_{\mathrm{std}}),\\
\xi &\colon X\to A,\quad x\mapsto x.
\end{aligned}
\]
Then \(X\) is compact, Hausdorff, totally disconnected, and has a basis of clopen subsets.
Hence \(\xi\in C_c(X,A)\) and \(\xi(X)=X\) is infinite.
Therefore \(\xi\notin \operatorname{im}(\Phi_X)\), so \(\Phi_X\) is not surjective.
\end{frame}

\begin{frame}{Non-discrete Coefficients: What Fails}{Isomorphism without discreteness.}
\[
A \coloneqq (\mathbb{R},\mathcal{O}_{\mathrm{std}}),
\qquad
(\{\bullet\}, \mathcal{O}_{\{\bullet\}}\coloneq\{\varnothing,\{\bullet\}\}).
\]
Then $\{\bullet\}$ is locally compact, totally disconnected, Hausdorff, and $X$ is compact open.
\[
\begin{aligned}
C_c(\{\bullet\},\mathbb{Z}) &\cong \mathbb{Z},\\
C_c(\{\bullet\},A) &\cong A,\\
C_c(\{\bullet\},\mathbb{Z})\otimes_{\mathbb{Z}}A &\cong \mathbb{Z}\otimes_{\mathbb{Z}}A\cong A.
\end{aligned}
\]
Under these identifications the canonical map
\[
\Phi_{\{\bullet\}}:\ C_c(\{\bullet\},\mathbb{Z})\otimes_{\mathbb{Z}}A\to C_c(\{\bullet\},A),
\qquad
\chi_{\{\bullet\}}\otimes a\mapsto a\cdot \chi_{\{\bullet\}},
\]
is the standard isomorphism $\mathbb{Z}\otimes_{\mathbb{Z}}A\to A$, $1\otimes a\mapsto a$.
\end{frame}

\begin{frame}[plain,c]
  \begin{center}
    \Huge \textbf{Proof of Moore--Mayer--Vietoris}
  \end{center}
\end{frame}

\begin{frame}{Proof of Moore--Mayer--Vietoris}{Proof idea for \(H_n(\alpha_\bullet)\).}
  \((\iota_i)_n\colon (\G|_{U_{12}})_n\hookrightarrow (\G|_{U_i})_n\) is an open embedding, hence a local homeomorphism.
  Therefore the functorial pushforward on Moore chains
  \[
    ((\iota_i)_n)_*\colon
    C_c\bigl((\G|_{U_{12}})_n,A\bigr)\to C_c\bigl((\G|_{U_i})_n,A\bigr)
  \]
  is given by a \(\highlight[red]{\text{finite fibre sum}}\).
  Since \((\iota_i)_n\) is injective, it is extension by zero:
  \[
    ((\iota_i)_n)_* f(x)\coloneqq
    \begin{cases}
      f(x), & x\in (\G|_{U_{12}})_n,\\
      0, & x\in (\G|_{U_i})_n\setminus (\G|_{U_{12}})_n.
    \end{cases}
  \]
  \pause

  Define the chain map
  \[
    \begin{aligned}
      \alpha_n\colon C_c\bigl((\G|_{U_{12}})_n,A\bigr)
      &\to
      C_c\bigl((\G|_{U_1})_n,A\bigr)\oplus C_c\bigl((\G|_{U_2})_n,A\bigr),\\
      f &\mapsto \bigl(((\iota_1)_n)_* f,\,-((\iota_2)_n)_* f\bigr).
    \end{aligned}
  \]
  \(\highlight[blue]{\text{Compatibility:}}\)
  \(
    \partial\,((\iota_i)_n)_* = ((\iota_i)_{n-1})_*\,\partial,
    \partial\,\alpha_n = \alpha_{n-1}\,\partial.
  \)
  Hence \(\alpha_\bullet\) induces \(H_n(\alpha_\bullet)\).
\end{frame}

\begin{frame}{Proof of Moore--Mayer--Vietoris}{Proof idea for \(H_n(\beta_\bullet)\).}
  Let \(\kappa_i\colon \G|_{U_i}\hookrightarrow \G\) be the inclusion of reductions.

  \((\kappa_i)_n\colon (\G|_{U_i})_n\hookrightarrow \G_n\) is an open embedding, hence a local homeomorphism.
  Therefore the pushforward on Moore chains extends by zero:
  \[
    ((\kappa_i)_n)_* g(x)\coloneqq
    \begin{cases}
      g(x), & x\in (\G|_{U_i})_n,\\
      0, & x\in \G_n\setminus (\G|_{U_i})_n.
    \end{cases}
  \]

  Define
  \[
    \begin{aligned}
      \beta_n\colon
      C_c\bigl((\G|_{U_1})_n,A\bigr)\oplus C_c\bigl((\G|_{U_2})_n,A\bigr)
      &\to C_c(\G_n,A),\\
      (g_1,g_2) &\mapsto ((\kappa_1)_n)_* g_1 + ((\kappa_2)_n)_* g_2.
    \end{aligned}
  \]
  \(\highlight[blue]{\text{Compatibility:}}\)
  \(
    \partial\,((\kappa_i)_n)_* = ((\kappa_i)_{n-1})_*\,\partial,
    \partial\,\beta_n = \beta_{n-1}\,\partial.
  \)
  Hence \(\beta_\bullet\) induces \(H_n(\beta_\bullet)\).
\end{frame}

\begin{frame}{Proof of Moore--Mayer--Vietoris}{Proof idea for \(\partial_n\).}
  Assume a \(\highlight[red]{\text{SES}}\) of Moore chain complexes
  \[
    \begin{aligned}
      0 \to C_c\bigl((\G|_{U_{12}})_\bullet,A\bigr)
      &\xrightarrow{\alpha_\bullet}
      C_c\bigl((\G|_{U_1})_\bullet,A\bigr)\oplus C_c\bigl((\G|_{U_2})_\bullet,A\bigr)
      \\
      &\xrightarrow{\beta_\bullet}
      C_c\bigl(\G_\bullet,A\bigr)\to 0.
    \end{aligned}
  \]
  Here \(\partial\) denotes the Moore boundary.

  Let \([c]\in H_n(\G;A)\) with \(\partial c=0\) and choose \(b\) with \(\highlight[red]{\beta_n(b)=c}\).

  Then \(\beta_{n-1}(\partial b)=\partial(\beta_n(b))=\partial c=0\), hence
  \[
    \partial b\in \ker(\beta_{n-1})=\operatorname{im}(\alpha_{n-1}).
  \]
  Choose \(a\in C_c\bigl((\G|_{U_{12}})_{n-1},A\bigr)\) with \(\highlight[red]{\alpha_{n-1}(a)=\partial b}\) and define
  \[
    \partial_n([c])\coloneqq [a]\in H_{n-1}(\G|_{U_{12}};A).
  \]

  \(\highlight[blue]{\text{Standard homological algebra:}}\)
  \(\partial_n\) is well-defined, independent of choices, and yields exactness at \(H_n(\G;A)\).
\end{frame}

\begin{frame}[allowframebreaks, noframenumbering]{References}
  \printbibliography
  \nocite{*}
\end{frame}

\end{document}